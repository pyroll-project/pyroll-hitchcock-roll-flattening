%! Author = Christoph Renzing
%! Date = 21.04.22

% Preamble
\documentclass[11pt]{PyRollDocs}
\usepackage{textcomp}

\addbibresource{refs.bib}
% Document
\begin{document}

    \title{Hitchcock roll flattening PyRolL Plugin}
    \author{Christoph Renzing}
    \date{\today}

    \maketitle

    This plugin provides the analytical roll flattening model developed by \textcite{Hitchcook1935} and adapted by \textcite{BohmFlaxa1981, FlaxaHinkfothBohm1979}.


    \section{Model approach}\label{sec:model-approach}

    The models are derived from the general theory of elasticity.
    According to Hitchcock, roll flattening due too high forces and large contact lengths has to be considered as it enhances calculation of roll force and torque.
    \textcite{Hitchcook1935} assumed the pressure distribution for rolling to be elliptical,
    therefore the roll shape stays cylindrical and a replacement radius larger than the nominal radius can be calculated.
    Bohm and Flaxa \textcite{BohmFlaxa1981, FlaxaHinkfothBohm1979} extended the usage of the method to greater initial ($R_0$) to flattened roll radii ($R_1$)
    For the calculation the roll force and the elastic constants of the roll material are required.
    Furthermore, the calculation has to be done in an iterative way.
    To implement the method for groove rolling, the calculations are done using a equivalent rectangle.
    Equivalent variables are denoted $eq$.

    To calculate the flattened radius following equation is used:

    \begin{subequations}
        \begin{equation}
            \frac{R_\mathrm{1}}{R_\mathrm{0}} = 1 + \frac{16}{\pi} \frac{ - \nu_\mathrm{W}^2}{E_\mathrm{W}} \frac{F_\mathrm{Roll}}{h_\mathrm{eq,0} - s}, \frac{R_\mathrm{1}}{R_\mathrm{0}} < 5.235
            \label{eq:hitchcook-1}\\
        \end{equation}
        \begin{equation}
            \frac{R_\mathrm{1}}{R_\mathrm{0}} = \left(  \frac{16}{\pi} \frac{1 - \nu_\mathrm{W}^2}{E_\mathrm{W}} \frac{F_\mathrm{Roll}}{h_\mathrm{eq,0} - s} \right)^{\frac{2}{3}}, \frac{R_\mathrm{1}}{R_\mathrm{0}} > 5.235
            \label{eq:hitchcook-2}
        \end{equation}
    \end{subequations}


    \section{Usage instructions}\label{sec:usage-instructions}

    The plugin can be loaded under the name \texttt{pyroll\_hitchcock\_roll\_flattening}.

    An implementation of the hooks \lstinline{flattened_radius} and \lstinline{flattening_ratio}
    for calculation the values for $R_1$ and $\frac{R_1}{R_0}$ are provided on the \lstinline{Roll}.
    The hooks \lstinline{elastic_modulus} as well as \lstinline{poissons_ratio} have to be set by the user on \lstinline{Roll}.
    Several additional hooks on \lstinline{Roll} are defined, which are used for calculation, as listed in \autoref{tab:hookspecs}.

    \begin{table}
        \centering
        \caption{Hooks specified by this plugin.}
        \label{tab:hookspecs}
        \begin{tabular}{ll}
            \toprule
            Hook name                      & Meaning                                                      \\
            \midrule
            \texttt{poissons\_ratio}       & Poissons's ratio of roll material $\nu_\mathrm{W}$           \\
            \texttt{youngs\_modulus}       & Youngs's modulus of roll material $E_\mathrm{W}$             \\
            \texttt{flattening\_ratio}     & Ratio between flattened and initial radius $\frac{R_1}{R_0}$ \\
            \texttt{flattened\_radius}     & Flattened roll radius $R_1$                                  \\
            \texttt{max\_roll\_radius}     & Max.~roll radius                                             \\
            \texttt{min\_roll\_radius}     & Min.~roll radius                                             \\
            \texttt{working\_roll\_radius} & Working roll radius                                          \\
            \bottomrule
        \end{tabular}
    \end{table}

    \printbibliography


\end{document}